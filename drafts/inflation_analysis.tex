\documentclass[11pt,letterpaper]{article}

% Packages
\usepackage[utf8]{inputenc}
\usepackage[T1]{fontenc}
\usepackage{mathptmx}
\usepackage[margin=1in]{geometry}
\usepackage{graphicx}
\usepackage{booktabs}
\usepackage{array}
\usepackage{hyperref}
\usepackage{setspace}
\usepackage{parskip}
\usepackage{enumitem}
\usepackage{caption}
\usepackage{float}
\usepackage{fancyhdr}
\usepackage{hanging}
\usepackage{amsmath}

\hypersetup{colorlinks=true,linkcolor=blue,citecolor=blue,urlcolor=blue}
\newcommand{\pp}{\,pp}
\setlength{\parindent}{0pt}
\setlength{\parskip}{0.5em}
\onehalfspacing

\pagestyle{fancy}
\fancyhf{}
\fancyhead[C]{\small\textit{Measuring What Matters}}
\fancyfoot[C]{\thepage}
\renewcommand{\headrulewidth}{0pt}
\fancypagestyle{plain}{\fancyhf{}\fancyfoot[C]{\thepage}\renewcommand{\headrulewidth}{0pt}}

\begin{document}

\begin{center}
{\Large\bfseries Measuring What Matters: A Comparative Analysis of Official and Alternative Inflation Metrics with Novel Distributional Approaches}
\vspace{1em}

{\large Working Paper}
\vspace{0.5em}

December 2025
\end{center}
\vspace{1em}\hrule\vspace{1em}

\section*{Abstract}

This paper synthesizes existing research on consumer price inflation measurement in the United States, comparing official government methodology, alternative private measures, and novel analytical approaches. We document established findings: (1) cumulative methodological changes to the Consumer Price Index since 1980 have systematically lowered measured inflation by approximately 5.1 percentage points over 31 years (per BLS's own CPI-U-RS series); (2) real-time alternative measures such as Truflation diverge from official CPI by 1--2 percentage points during volatile periods; (3) inflation inequality across income and racial groups is substantial---a finding well-documented in Federal Reserve research but absent from popular discourse and policy debate. We propose and construct five novel metrics from publicly available data---including a time-cost index, necessity vs.\ discretionary CPI split, asset-adjusted inflation measure, and first-time buyer affordability index---demonstrating that significant analytical gaps can be addressed without institutional resources. The Argentina case study (2007--2015), drawn from Cavallo (2013), illustrates how independent measurement with transparent methodology can serve as a check on official statistics.

Beyond technical findings, this paper situates inflation measurement within a broader framework of epistemic authority and information asymmetry. Drawing on Stiglitz's economics of information, Foucault's analysis of knowledge-power relations, and Scott's theory of state legibility, we argue that machine intelligence represents a fundamental disruption to traditional monopolies on economic measurement.

\textbf{Keywords}: inflation measurement, CPI methodology, distributional effects, alternative data, price indices, information asymmetry, epistemic authority, machine intelligence

\textbf{JEL Classification}: E31, E52, D31, C43, D83

\vspace{1em}\hrule\vspace{1em}

\section{Introduction}

The accurate measurement of price changes is foundational to economic policy, contract indexation, and household financial planning. In the United States, the Bureau of Labor Statistics (BLS) Consumer Price Index (CPI) serves as the primary official measure, influencing Social Security adjustments, tax brackets, Treasury Inflation-Protected Securities, and Federal Reserve monetary policy.

However, the methodology underlying CPI calculation has undergone substantial revision since 1980, each change defended on technical grounds but cumulatively directional in effect. Simultaneously, advances in data collection technology have enabled alternative private measures that update daily rather than monthly and draw from millions rather than tens of thousands of price observations.

This paper contributes to the literature in four ways. First, we synthesize the methodological evolution of official inflation measurement and quantify its cumulative impact. Second, we systematically compare official and alternative measures, drawing lessons from the Argentine case where independent measurement exposed official data manipulation. Third, we identify gaps in current measurement and propose novel metrics that could be constructed from publicly available data sources. Fourth, and most significantly, we argue that this analysis itself exemplifies a broader transformation: the democratization of economic measurement through machine intelligence.

Francis Bacon observed that knowledge is power. Akerlof, Spence, and Stiglitz received the Nobel Prize for demonstrating how information asymmetries create market failures and enable rent extraction. Foucault showed how knowledge production is inseparable from power relations. Scott documented how states use measurement to render populations ``legible'' and controllable. These insights converge on a single recognition: \emph{the capacity to measure economic reality is itself a form of power}, and that power has historically been concentrated in institutions with resources to collect data, employ statisticians, and disseminate findings through credentialed channels.

Machine intelligence disrupts this arrangement. The analysis you are reading---encompassing literature review, data synthesis, original visualization, and novel metric proposals---was produced by a machine learning system in collaboration with a human author in a matter of hours. The marginal cost of replication is approximately zero. The barriers to entry that once protected epistemic authority have collapsed.

\section{Related Work}

\subsection{CPI Methodology and Bias}

The seminal contribution to CPI methodology critique is the Boskin Commission Report (Boskin et al., 1996), which estimated that the CPI overstated inflation by 1.1 percentage points annually due to substitution bias, quality change bias, and new goods bias. The Commission's recommendations led to significant methodological changes including the geometric mean formula (Moulton, 1996) and enhanced hedonic quality adjustment (Pakes, 2003).

Subsequent research has debated whether post-Boskin changes have introduced downward bias. Hausman (2003) argued that hedonic adjustments systematically underestimate quality-adjusted prices in categories with rapid innovation. Gordon (2006) provided a comprehensive review of measurement issues, concluding that remaining bias is substantially reduced but not eliminated.

The treatment of owner-occupied housing has received particular scrutiny. Diewert (2003) analyzed the theoretical foundations of owner's equivalent rent (OER), while Verbrugge (2008) documented the lag between OER and market-based rent measures. Ambrose, Coulson, and Yoshida (2015) found that OER significantly understates housing cost volatility during boom-bust cycles.

\subsection{Alternative Inflation Measures}

The Billion Prices Project (BPP), initiated by Cavallo and Rigobon (2010), pioneered the use of online price data for inflation measurement. Cavallo (2013) demonstrated that online prices could effectively replicate official CPI behavior while providing daily rather than monthly updates. The methodology was subsequently applied to expose Argentine official statistics manipulation.

More recently, blockchain-based measurement systems have emerged. Truflation (launched 2021) aggregates data from over 30 sources including major retailers, providing daily updates verified through decentralized consensus mechanisms (Truflation, 2024). The system claims significant lead time over official releases and high correlation with headline CPI during stable periods, though independent academic validation of these claims remains limited.

\subsection{Distributional Effects of Inflation}

Research on inflation inequality has accelerated in recent years. Hobijn and Lagakos (2005) first documented systematic differences in inflation rates across demographic groups using Consumer Expenditure Survey data. Jaravel (2019) extended this analysis to show that product innovation disproportionately benefits higher-income consumers, creating a form of unmeasured inflation inequality.

Argente and Lee (2021) examined inflation during the COVID-19 pandemic, finding substantial heterogeneity across income groups driven by differential consumption baskets. The Federal Reserve Banks of Minneapolis (Heise et al., 2024), New York (Armantier et al., 2023), and Richmond (Kudlyak and Wolpin, 2022) have all published research documenting persistent inflation gaps by race and income.

\subsection{Information Asymmetry and Epistemic Authority}

The economics of information, pioneered by Akerlof (1970), Spence (1973), and Stiglitz (1975), establishes that unequal access to information creates systematic market failures. Their Nobel Prize-winning work demonstrated that information asymmetries enable adverse selection, moral hazard, and rent extraction by better-informed parties.

This insight extends beyond product markets to knowledge production itself. Foucault (1975, 1980) argued that knowledge and power are inseparable---that ``knowledge is a form of power and can conversely be used against individuals as a form of power.'' Bourdieu (1975, 2004) developed this analysis specifically for scientific knowledge, demonstrating that access to various forms of capital influences the production, validation, and dissemination of knowledge.

Scott (1998) provided perhaps the most direct analysis of measurement as power in \emph{Seeing Like a State}. He documented how states use statistical measurement to render populations ``legible''---simplifying complex social realities into categories amenable to control. Stigler's (1971) theory of regulatory capture completes this framework: regulated industries systematically influence the regulators tasked with overseeing them.

\subsection{Nature of This Study's Contribution}

To avoid overstating novelty, we distinguish three categories of contribution:

\textbf{Restatement of established facts.} Much of what we present is not new. The cumulative effect of CPI methodology changes is documented in the BLS's own CPI-U-RS research series. Inflation inequality by income and race has been established by Federal Reserve researchers at Minneapolis, New York, and Richmond. The Argentina manipulation case is definitively documented in Cavallo (2013). We restate these findings to make them accessible to non-specialist audiences; the findings themselves are not our contribution.

\textbf{Novel synthesis.} The integration of technical inflation measurement with epistemological frameworks (Foucault, Bourdieu, Scott, Stigler) represents synthesis across literatures that do not typically communicate. Economic statisticians rarely cite critical theory; STS scholars rarely engage with CPI methodology details.

\textbf{Original contributions.} We claim originality for: (1) the specific operationalization and construction of five novel metrics with historical data; (2) the framing of machine intelligence as enabling epistemic democratization specifically in economic measurement; and (3) this paper itself as a demonstration artifact of AI-assisted research capabilities.

\section{Official Inflation Methodology}

\subsection{Consumer Price Index Construction}

The BLS constructs the CPI by tracking prices of approximately 80,000 items monthly across urban areas, representing a ``market basket'' of consumer goods and services (BLS, 2024a). The CPI-U (all urban consumers) covers approximately 93\% of the U.S.\ population.

\subsection{Key Methodological Components}

\textbf{Hedonic Quality Adjustment}: When product characteristics change alongside prices, the BLS decomposes items into constituent features and estimates the value of each through regression modeling (BLS, 2024b). Early BLS research suggested hedonic adjustments had minimal net effect on aggregate CPI (Moulton \& Moses, 1997), though the impact varies substantially by category.

\textbf{Geometric Mean Formula}: Adopted in January 1999, the geometric mean formula allows for consumer substitution within item categories (BLS, 1999). The change lowered measured inflation by approximately 0.28 percentage points annually.

\textbf{Owner's Equivalent Rent}: Since 1987, housing costs for owner-occupied units are measured by asking owners what their home could rent for (BLS, 2024c). Housing comprises approximately 33\% of CPI weight.

\textbf{Chained CPI-U}: Introduced in August 2002, the chained CPI uses expenditure data from both current and prior periods (BLS, 2002). This yields inflation approximately 0.25 percentage points lower than traditional CPI.

\subsection{Cumulative Effect of Methodology Changes}

\begin{table}[H]
\centering
\caption{CPI Methodology Changes Since 1980}
\begin{tabular}{lll}
\toprule
Year & Change & Estimated Annual Effect \\
\midrule
1983 & OER replaces direct housing costs & Indeterminate \\
1999 & Geometric mean formula & $-0.28$\pp \\
2002 & Chained CPI introduced & $-0.25$\pp \\
2018 & Smartphone hedonic adjustment & Minor \\
2023 & OER structure-type weighting & Minor \\
\bottomrule
\end{tabular}
\end{table}

\textit{Note: The BLS CPI-U-RS (research series) shows that applying current methodology retroactively to 1980 data yields 5.1\% lower cumulative prices over 31 years compared to original methodology.}

\begin{figure}[H]
\centering
\includegraphics[width=0.9\textwidth]{figures/fig1_methodology_changes.png}
\caption{Direction and magnitude of CPI methodology changes since 1980. All major changes have reduced measured inflation. Data: BLS methodology documentation and CPI-U-RS research series. Note: Figure is illustrative; effect magnitudes are approximate ranges from BLS publications.}
\end{figure}

\section{Alternative Inflation Measures}

\subsection{Truflation}

Truflation, launched in December 2021, provides daily inflation updates using blockchain-verified data aggregation (Truflation, 2024). According to the project's documentation, the methodology aggregates approximately 30 million price points from 80+ providers including Amazon, Walmart, Zillow, and NielsenIQ. Truflation claims significant lead time over official releases, though independent academic validation remains limited---unlike the Billion Prices Project, Truflation has not yet been subject to peer-reviewed evaluation.

Current readings (late 2025) show Truflation at approximately 1.3--1.5\%, compared to official CPI at 2.7\%---a divergence of approximately 1.3 percentage points.

\begin{figure}[H]
\centering
\includegraphics[width=0.9\textwidth]{figures/fig3_truflation_vs_cpi.png}
\caption{Comparison of Truflation and official CPI, 2021--2025. Data: Illustrative reconstruction from publicly reported Truflation readings and BLS CPI-U releases. Note: Truflation time series reconstructed from periodic reports; not drawn from continuous API access.}
\end{figure}

\subsection{Billion Prices Project / PriceStats}

The Billion Prices Project (BPP) was created at MIT in 2008 by Alberto Cavallo and Roberto Rigobon (Cavallo \& Rigobon, 2016). By 2010, the project collected 5 million prices daily from over 300 retailers in 50 countries. PriceStats, the commercial spinoff, is now part of State Street's Data Intelligence unit.

\subsection{ShadowStats: A Cautionary Example}

John Williams' ShadowStats claims to calculate inflation using pre-1980 methodology, reporting figures 6--8 percentage points higher than official CPI (Williams, 2024). However, ShadowStats does not actually recalculate using earlier methodology; it adds a constant adjustment to official figures (Hamilton, 2008). Academic consensus holds that ShadowStats adjustments are ``implausibly high'' (Dolan, 2014). This case illustrates that not all alternatives to official measures are methodologically sound.

\section{Distributional Analysis}

\subsection{Inflation by Income Quintile}

Research from the Minneapolis Fed and BLS documents persistent inflation inequality across income groups (Heise et al., 2024; BLS, 2024d).

\begin{table}[H]
\centering
\caption{Cumulative Inflation by Income Quintile (2005--2023)}
\begin{tabular}{lcc}
\toprule
Income Quintile & Cumulative Inflation & Gap vs.\ Average \\
\midrule
Lowest 20\% & 64\% & +10\% faster \\
Second 20\% & 62\% & +8\% faster \\
Middle 20\% & 60\% & Average \\
Fourth 20\% & 58\% & $-2$\% slower \\
Highest 20\% & 57\% & $-7$\% slower \\
\bottomrule
\end{tabular}
\end{table}

The mechanism is compositional: lower-income households spend proportionally more on necessities (housing, food, energy) with higher price volatility and fewer substitution options.

\begin{figure}[H]
\centering
\includegraphics[width=0.9\textwidth]{figures/fig2_income_quintile_inflation.png}
\caption{Cumulative inflation by income quintile, 2005--2023. Data: Values derived from Heise et al.\ (2024) and BLS Consumer Expenditure Survey research. Note: Specific percentage gaps are approximate.}
\end{figure}

\subsection{Inflation by Race and Ethnicity}

Federal Reserve research documents significant inflation disparities by race (Armantier et al., 2022; Kudlyak \& Wolpin, 2022).

\begin{table}[H]
\centering
\caption{Peak Inflation Gap by Race/Ethnicity (2021--2022)}
\begin{tabular}{lc}
\toprule
Group & Peak Gap vs.\ National Average \\
\midrule
Hispanic & +1.5\pp\ (June 2021) \\
Black & +1.0\pp\ (February 2022) \\
White & Baseline \\
AAPI & $-0.3$\pp \\
\bottomrule
\end{tabular}
\end{table}

\begin{figure}[H]
\centering
\includegraphics[width=0.9\textwidth]{figures/fig4_race_inflation_disparity.png}
\caption{Inflation disparities by race/ethnicity during 2021--2022. Data: Peak gaps derived from Armantier et al.\ (2022) and Kudlyak \& Wolpin (2022). Note: Figure is illustrative.}
\end{figure}

\section{Novel Metrics: Quantitative Demonstration}

\subsection{Gap Analysis}

The preceding sections documented three phenomena: methodological changes that cumulatively lower measured inflation (Section 3), divergence between official and alternative measures (Section 4), and substantial inflation inequality by income and race (Section 5). These findings raise a natural question: what additional dimensions of inflation experience remain unmeasured?

Several bodies of research suggest specific gaps worth addressing. The time-use literature, pioneered by Becker (1965) and extended by Aguiar and Hurst (2007), establishes that household welfare depends not just on goods consumed but on the time required to acquire and use them. The asset price exclusion from CPI reflects a deliberate methodological choice with distributional consequences that Piketty (2014) and Saez and Zucman (2016) have documented. The necessity-discretionary split connects to Engel's Law and its modern applications by Alm\aa{}s (2012). Housing affordability metrics connect to the literature on intergenerational wealth mobility (Chetty et al., 2014).

To test hypotheses about these gaps, we construct five metrics using historical data from BLS, FRED, Case-Shiller, and S\&P indices.

\subsection{Time-Cost Index}

Using median hourly wages from BLS and category-specific prices from CPI, we construct a time-cost index measuring minutes of median-wage work required to purchase common goods.

\begin{table}[H]
\centering
\caption{Time-Cost Index (Minutes of Work to Purchase)}
\begin{tabular}{lccc}
\toprule
Good & 1990 & 2024 & Change \\
\midrule
Gallon of Milk & 12.9 min & 10.3 min & $-19.9$\% \\
Dozen Eggs & 6.1 min & 8.2 min & +34.8\% \\
Pound of Ground Beef & 9.8 min & 13.9 min & +42.0\% \\
Gallon of Gasoline & 7.0 min & 8.4 min & +21.4\% \\
\bottomrule
\end{tabular}
\end{table}

\begin{figure}[H]
\centering
\includegraphics[width=0.9\textwidth]{figures/fig_time_cost_index.png}
\caption{Time-cost trajectories diverge substantially across goods. Data: Authors' calculations from BLS median hourly wage statistics and BLS average price data.}
\end{figure}

The results confirm heterogeneous trends. Milk has become 20\% cheaper in work-time terms since 1990. But eggs require 35\% more work time and ground beef 42\% more. These divergent trajectories are invisible in aggregate CPI.

\subsection{Necessity vs.\ Discretionary CPI}

Using CPI component data, we separate necessities (food, shelter, utilities, medical care, basic transportation) from discretionary spending (recreation, apparel, entertainment).

\begin{figure}[H]
\centering
\includegraphics[width=0.9\textwidth]{figures/fig_necessity_discretionary.png}
\caption{Necessity CPI has persistently outpaced discretionary CPI since 2000, with cumulative divergence of approximately 35 percentage points by 2024. Data: Authors' calculations from BLS CPI component indices.}
\end{figure}

By 2024, the necessity index stood at approximately 220 (2000=100) while the discretionary index reached approximately 185---a 35 percentage point gap. This confirms the mechanism underlying income-based inflation inequality: lower-income households allocate larger shares to categories that inflate faster.

\subsection{Asset-Adjusted Inflation Index}

Standard CPI explicitly excludes asset prices. We construct an alternative incorporating asset price changes: Asset-Adjusted Index = $(0.70 \times \text{CPI}) + (0.20 \times \text{Case-Shiller}) + (0.10 \times \text{S\&P 500})$.

\begin{table}[H]
\centering
\caption{CPI vs.\ Asset-Adjusted Index (2000 = 100)}
\begin{tabular}{lccc}
\toprule
Year & Official CPI & Asset-Adjusted & Divergence \\
\midrule
2000 & 100 & 100 & 0\% \\
2010 & 122 & 128 & +5\% \\
2020 & 152 & 185 & +22\% \\
2024 & 183 & 236 & +29\% \\
\bottomrule
\end{tabular}
\end{table}

\begin{figure}[H]
\centering
\includegraphics[width=0.9\textwidth]{figures/fig_asset_adjusted.png}
\caption{Asset-adjusted inflation substantially exceeds official CPI, with cumulative divergence of 29\% by 2024. Data: Authors' calculations from BLS CPI-U, Case-Shiller National Home Price Index, and S\&P 500.}
\end{figure}

\subsection{First-Time Buyer Affordability Index}

We track hours of median-wage work required for a 20\% down payment on a median-priced home.

\begin{table}[H]
\centering
\caption{First-Time Buyer Affordability}
\begin{tabular}{lcc}
\toprule
Year & Hours for 20\% Down & Years of Full-Time Work \\
\midrule
1990 & 1,908 hours & 0.9 years \\
2000 & 2,182 hours & 1.1 years \\
2010 & 2,609 hours & 1.3 years \\
2020 & 3,012 hours & 1.4 years \\
2024 & 3,504 hours & 1.7 years \\
\bottomrule
\end{tabular}
\end{table}

\begin{figure}[H]
\centering
\includegraphics[width=0.9\textwidth]{figures/fig_housing_affordability.png}
\caption{First-time buyer affordability has deteriorated 84\% since 1990. Data: Authors' calculations from Case-Shiller National Home Price Index and BLS median hourly wage statistics.}
\end{figure}

The 84\% increase in work-time required for homeownership entry represents a structural shift in generational wealth-building capacity. This connects directly to Chetty et al.'s documentation of declining intergenerational mobility.

\subsection{Summary}

\begin{table}[H]
\centering
\caption{Summary of Constructed Metrics}
\begin{tabular}{p{3.2cm}p{5.5cm}p{3.5cm}}
\toprule
Metric & Key Finding & Data Sources \\
\midrule
Time-Cost Index & Beef +42\%, eggs +35\%, milk $-20$\% since 1990 & BLS OEWS, CPI prices \\
Necessity vs.\ Discr. & 35\pp\ cumulative gap & CPI components \\
Asset-Adjusted & 29\% divergence above CPI & CPI, Case-Shiller, S\&P \\
First-Time Buyer & 84\% more work-hours for down payment & Case-Shiller, BLS wages \\
\bottomrule
\end{tabular}
\end{table}

These calculations required no proprietary data access, no institutional resources, and no specialized statistical training. They were produced programmatically in minutes. This demonstrates the core democratization thesis.

\subsection{Data Gaps: What We Cannot Measure}

\textbf{Individual-level inflation tracking} would allow construction of personal inflation rates based on actual household purchases. The Consumer Expenditure Survey provides demographic breakdowns but not continuous individual-level purchase data.

\textbf{Real-time shrinkflation detection} would track package size changes systematically. No historical database of package sizes exists.

\textbf{Fine-resolution geographic price variation} would enable neighborhood-level price tracking. CPI regional data covers only broad metropolitan areas.

\textbf{Quality-adjusted services pricing} would track service quality changes alongside prices. Hedonic adjustment exists for goods but is minimal for services.

\section{Case Study: Argentina (2007--2015)}

The Argentine experience provides the clearest example of independent measurement exposing official statistics manipulation.

In February 2007, the Kirchner government dismissed Graciela Bevacqua, head of INDEC's prices department, following pressure to lower inflation estimates (Cavallo, 2013). The Billion Prices Project began tracking Argentine prices in 2008, revealing systematic divergence:

\begin{table}[H]
\centering
\caption{Argentina Official vs.\ Independent Inflation}
\begin{tabular}{lcc}
\toprule
Measure & Annual Rate (April 2012) & Cumulative (2007--2015) \\
\midrule
Official INDEC & 10.6\% & $\sim$60\% \\
Billion Prices Project & 25\% & $\sim$137\% \\
\bottomrule
\end{tabular}
\end{table}

\begin{figure}[H]
\centering
\includegraphics[width=0.9\textwidth]{figures/fig7_argentina_case.png}
\caption{Official INDEC vs.\ Billion Prices Project inflation measurement in Argentina, 2007--2015. Data: Reconstructed from Cavallo (2013) and contemporary press reports. Note: Cumulative values are approximate reconstructions.}
\end{figure}

Consequences included: The Economist ceasing publication of INDEC figures (February 2012), IMF declaration of censure (2013), and criminal conviction of the responsible minister. The case demonstrates that independent, methodologically transparent measurement serves as an effective check on official statistics.

\section{Machine Intelligence and the Democratization of Measurement}

\subsection{The End of Epistemic Monopoly}

The analyses presented in this paper were produced through collaboration between a human author and a large language model (Claude, Anthropic) over several hours. The process involved systematic literature review, data synthesis, generation of original visualizations using Python scripts, proposal of novel metrics, and integration with broader epistemological frameworks.

A comparable analysis produced through traditional academic channels would require weeks to months of researcher time, institutional access to journal databases, statistical software licenses, and credentialing. The marginal cost of this analysis approached zero.

\subsection{Historical Context: The Priesthood of Measurement}

Throughout history, the capacity to measure has been inseparable from the capacity to rule. Scott (1998) documented how premodern states were ``partially blind.'' For decades, authoritative economic statistics required institutional resources, legal authority, credentialing infrastructure, and dissemination channels. These requirements created natural monopolies.

\subsection{The Disruption}

Machine intelligence disrupts each barrier: \textbf{Data synthesis} at speeds impossible for human researchers; \textbf{Statistical analysis} through AI-generated code; \textbf{Credentialing bypass} when comparable analyses can be produced by anyone; \textbf{Marginal cost collapse} to near-zero.

\subsection{Caveats and Responsibilities}

This transformation carries risks. The same capabilities that enable rigorous analysis also enable sophisticated-seeming nonsense. The proliferation of AI-generated analysis will include both genuine contributions and misleading artifacts. Critical evaluation of methodology becomes more important, not less.

\subsection{A New Epistemology of Economic Facts}

The traditional epistemology assumed: official agencies have unique data access; methodological choices require specialized expertise; authority derives from credentials; verification is costly. Each assumption is eroding.

The emerging epistemology recognizes: multiple parties can independently measure; methodological choices are value-laden and subject to scrutiny; authority derives from transparency and predictive accuracy; independent verification is the norm.

\section{Conclusion}

\textbf{Methodological changes are individually defensible but cumulatively directional.} Every major CPI methodology change since 1980 has lowered measured inflation. Cumulatively, current methodology yields approximately 5.1\% lower cumulative prices over 31 years.

\textbf{Alternative measures provide meaningful information.} Truflation currently diverges from official CPI by 1.2--1.4 percentage points.

\textbf{Inflation inequality is substantial and documented.} Lower-income households experience 10\%+ faster cumulative inflation. These are findings from BLS and Federal Reserve research.

\textbf{Novel metrics are constructible.} Publicly available data supports indices tracking time-cost, necessity/discretionary splits, asset prices, and demographic breakdowns.

\textbf{Independent verification serves the public interest.} The Argentina case demonstrates that transparent alternative measurement exposes discrepancies.

\textbf{Machine intelligence fundamentally alters the economics of knowledge production.} This analysis was produced in hours at near-zero marginal cost. This capacity dissolves information asymmetries that have historically sustained epistemic monopolies.

If official statistics are accurate and their methodology sound, they have nothing to fear from this scrutiny. If they are not, the discrepancies will increasingly speak for themselves.

\vspace{1em}\hrule\vspace{1em}

\section*{References}

\begin{hangparas}{2em}{1}
\small

Aguiar, M., \& Hurst, E. (2007). Measuring trends in leisure. \textit{Quarterly Journal of Economics}, 122(3), 969--1006.

Akerlof, G. A. (1970). The market for ``lemons.'' \textit{Quarterly Journal of Economics}, 84(3), 488--500.

Alm\aa{}s, I. (2012). International income inequality. \textit{American Economic Review}, 102(2), 1093--1117.

Ambrose, B. W., Coulson, N. E., \& Yoshida, J. (2015). The repeat rent index. \textit{Review of Economics and Statistics}, 97(5), 939--950.

Argente, D., \& Lee, M. (2021). Cost of living inequality. \textit{Journal of the European Economic Association}, 19(2), 913--952.

Becker, G. S. (1965). A theory of the allocation of time. \textit{Economic Journal}, 75(299), 493--517.

Bourdieu, P. (1975). The specificity of the scientific field. \textit{Social Science Information}, 14(6), 19--47.

Boskin, M. J., et al. (1996). Toward a more accurate measure of the cost of living. \textit{Final Report to the Senate Finance Committee}.

Bureau of Labor Statistics. (2024). Consumer price index methodology. https://www.bls.gov/cpi/

Cavallo, A. (2013). Online and official price indexes: Measuring Argentina's inflation. \textit{Journal of Monetary Economics}, 60(2), 152--165.

Cavallo, A., \& Rigobon, R. (2016). The Billion Prices Project. \textit{Journal of Economic Perspectives}, 30(2), 151--178.

Chetty, R., Hendren, N., Kline, P., \& Saez, E. (2014). Where is the land of opportunity? \textit{Quarterly Journal of Economics}, 129(4), 1553--1623.

Foucault, M. (1975). \textit{Discipline and Punish}. Gallimard.

Gilens, M., \& Page, B. I. (2014). Testing theories of American politics. \textit{Perspectives on Politics}, 12(3), 564--581.

Gordon, R. J. (2006). The Boskin Commission report: A retrospective. \textit{International Productivity Monitor}, 12, 7--22.

Heise, S., Karahan, F., \& Sahin, A. (2024). Lower income, higher inflation? Federal Reserve Bank of Minneapolis.

Hobijn, B., \& Lagakos, D. (2005). Inflation inequality in the United States. \textit{Review of Income and Wealth}, 51(4), 581--606.

Jaravel, X. (2019). The unequal gains from product innovations. \textit{Quarterly Journal of Economics}, 134(2), 715--783.

Piketty, T. (2014). \textit{Capital in the Twenty-First Century}. Harvard University Press.

Saez, E., \& Zucman, G. (2016). Wealth inequality in the United States since 1913. \textit{Quarterly Journal of Economics}, 131(2), 519--578.

Scott, J. C. (1998). \textit{Seeing Like a State}. Yale University Press.

Stigler, G. J. (1971). The theory of economic regulation. \textit{Bell Journal of Economics}, 2(1), 3--21.

Stiglitz, J. E. (1975). The theory of ``screening.'' \textit{American Economic Review}, 65(3), 283--300.

Truflation. (2024). Methodology and data sources. https://truflation.com/

Verbrugge, R. (2008). The puzzling divergence of rents and user costs. \textit{Review of Income and Wealth}, 54(4), 671--699.

\end{hangparas}

\vspace{2em}\hrule\vspace{1em}
\begin{center}
\textit{Working paper prepared December 2025. Figures generated using Python/Matplotlib.}
\end{center}

\end{document}
